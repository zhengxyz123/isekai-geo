\chapter{行星}
在本章中,我将先从一整个行星系开始,讲述行星系中所有行星、卫星的运行规律。然后,我将具体介绍一颗特定行星的运行规律。

本章涉及了一些天体物理的内容,它们可以更好地帮助我们创建异世界的架构,因为有些地理现象是由天体运动引起的,某些神话的主题也与天体有关。

\section{行星运动的规律}
在17世纪早期,德国天文学家开普勒发表了他著名的\textbf{开普勒定律}(Kepler's law),用三条相当简单的规则描述行星的运动。

虽然我们主要研究恒星与行星之间的关系,但卫星绕行星的运动同样符合开普勒定律。

\subsection{开普勒第一定律}
开普勒第一定律告诉我们:行星以椭圆轨道环绕着它们的恒星,而恒星位于椭圆的一个焦点上。

大部分行星轨道的离心率接近于0,如果对精度的要求不高,我们可以把椭圆近似为圆:恒星位于圆心,椭圆轨道的半长轴作为半径。

\subsection{开普勒第二定律}
开普勒第二定律告诉我们:在相等的时间内,恒星和运动着的行星的连线扫过的面积都相等。

由此可知,行星离恒星较近时,速度较快;离恒星较远时,速度较慢。

使用开普勒第二定律,我们可以计算出各个时刻行星的位置及速度。

若行星轨道的离心率接近于0,我们可以把椭圆近似为圆,则行星的角速度为\[\omega=\sqrt{\frac{GM}{a^3}}\]其中$G$是万有引力常数,$M$是中心天体的质量,$a$是轨道的半长轴(半径)。

若想更加精确地计算行星轨道,可使用圆锥曲线的极坐标方程\[\rho=\frac{ep}{1-e\cos\theta}\]其中$e$是椭圆的离心率,$p$是椭圆的准线到焦点的距离$\dfrac{a^2}{c}-c$,$\theta$为坐标原点与曲线上一点的连线与极轴所成角。

那么恒星和运动着的行星的连线在$t_1$时刻(对应的$\theta$为$\alpha$)和$t_2$时刻(对应的$\theta$为$\beta$)之间扫过的面积$A$为\[A=\int\nolimits_\alpha^\beta\frac{ep}{1-e\cos\theta}\;\mathrm{d}\theta\]

则根据开普勒第二定律得$\dfrac{\mathrm{d}A}{\mathrm{d}t}$应为定值。

\subsection{开普勒第三定律}
开普勒第三定律告诉我们:各个行星绕恒星的公转周期$T$的平方和轨道半长轴$a$的立方成正比,用公式表示为\[\frac{a^3}{T^2}=k\]其中$k$是常量,只与中心天体有关。

所以,距离恒星越远的行星,其公转周期就越长。

\section{行星的自转}

\section{行星的公转}
