\chapter{行星}
在本章中,我将先从一整个行星系开始,讲述行星系中所有行星、卫星的运行规律。然后,我将具体介绍一颗特定行星的运行规律。

本章涉及了一些天体物理的内容,它们可以更好地帮助我们创建异世界的架构,因为有些地理现象是由天体运动引起的,某些神话的主题也与天体有关。

\section{行星运动的规律}
在17世纪早期,德国天文学家开普勒发表了他著名的\textbf{开普勒定律}(Kepler's law),用三条相当简单的规则描述行星的运动。

虽然我们主要研究恒星与行星之间的关系,但卫星绕行星的运动同样符合开普勒定律。

\subsection{开普勒第一定律}
开普勒第一定律告诉我们:行星以椭圆轨道环绕着它们的恒星,而恒星位于椭圆的一个焦点上。

大部分行星轨道的离心率接近于0,如果对精度的要求不高,我们可以把椭圆近似为圆:恒星位于圆心,椭圆轨道的半长轴作为半径。

\subsection{开普勒第二定律}
开普勒第二定律告诉我们:在相等的时间内,恒星和运动着的行星的连线扫过的面积都相等。

由此可知,行星离恒星较近时,速度较快;离恒星较远时,速度较慢。

使用开普勒第二定律,我们可以计算出各个时刻行星的位置及速度。

若行星轨道的离心率接近于0,我们可以把椭圆近似为圆,则行星的角速度为\[\omega=\sqrt{\frac{GM}{a^3}}\]其中$G$是万有引力常数,$M$是中心天体的质量,$a$是轨道的半长轴(半径)。

若想更加精确地计算行星轨道,可使用圆锥曲线的极坐标方程\[\rho=\frac{ep}{1-e\cos\theta}\]其中$e$是椭圆的离心率,$p$是椭圆的准线到焦点的距离$\dfrac{a^2}{c}-c$,$\theta$为坐标原点与曲线上一点的连线与极轴所成角。

那么恒星和运动着的行星的连线在$t_1$时刻(对应的$\theta$为$\alpha$)和$t_2$时刻(对应的$\theta$为$\beta$)之间扫过的面积$A$为\[A=\int\nolimits_\alpha^\beta\frac{ep}{1-e\cos\theta}\;\mathrm{d}\theta\]

进一步,根据开普勒第二定律的推导\footnote{英语维基百科的Kepler's law of planetary motion条目中有详细的推导过程。},有\[\frac{\mathrm{d}A}{\mathrm{d}t}=\frac{\vec{r}\times\vec{p}}{2m}=\frac{\pi ab}{T}\]其中$\vec{r}\times\vec{p}$是行星的角动量,$m$是行星的质量,$a$、$b$分别是行星轨道的半长轴和半短轴,$T$是行星的周期。

\subsection{开普勒第三定律}
开普勒第三定律告诉我们:各个行星绕恒星的公转周期$T$的平方和轨道半长轴$a$的立方成正比,用公式表示为\[\frac{a^3}{T^2}=k\]其中$k$是常量,只与中心天体有关。

所以,距离恒星越远的行星,其公转周期就越长。

\section{行星的自转}
行星的自转会产生\textbf{科里奥利力}(coriolis force),其最典型的一种表现为\textbf{地转偏向力}。

如果把地球自转方向看成逆时针,那么北半球因为惯性受到顺时针的地转偏向力,使北半球的风向右偏转;南半球则相反,风会受到逆时针的地转偏向力而向左偏转;而在赤道上,因为介于南、北半球之间,地转偏向力则失效。 

地转偏向力有助于解释一些地理现象,在地球的北半球运动的物体(如河流、信风、气旋等)有向右偏转的趋势,在南半球运动的物体则有向左偏转的趋势。因此,北半球的河流右岸因侵蚀较强而多峭壁,左岸则多平缓河岸。南半球则反之。 

\section{行星的公转}
行星的公转会产生季节更替。地球由于拥有极小的轨道离心率(约为0.016)和$23^\circ26'$的转轴倾角,季节的更替是太阳直射点的周期性移动导致全球热量分配不均而引起的。这便可以解释为什么地球南、北半球的季节是相反的疑惑。如果行星的轨道离心率增大,那么恒星的热量在到达行星的时候会有不同程度的损失,导致季节现象更为明显。
