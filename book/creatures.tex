\chapter{生物}
在本章中,我将介绍无机环境与异世界生物之间的关系。生物从无机环境中诞生,生物活动时刻改变着无机环境,而变化的无机环境又反作用于生物本身。如此循环往复,最终达到平衡。

本章涉及了一些生物学,尤其是生态学的内容,这可以帮助我们更完美地描绘异世界的“外貌”。

\section{生物的演化}

\section{生态系统}

\subsection{种群}
在生态学上,\textbf{种群}(Population)是指在一定空间范围内同时生活的同种生物的全部个体。

\subsection{食物网}

\section{演替}
\textbf{演替}(Ecological succession)是一个群落的物种结构随时间变化的过程,是由低级到高级、由简单到复杂、一个阶段接着一个阶段、一个群落代替另一个群落的自然演变现象。可分为初生演替和次生演替两大类。

\subsection{初生演替}
\textbf{初生演替}(Primary succession)是一个“巨岩变细石,石上长青苔”的过程,会经历漫长的岁月。

\subsection{次生演替}
与初生演替不同,\textbf{次生演替}(Secondary succession)始于一场突发事件(如野火、飓风),致使原有群落中极度减少,但土壤并未受到十分严重的破坏的环境中。而初生演替的群落则缺乏土壤。
