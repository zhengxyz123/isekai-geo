\chapter{生物}
在本章中,我将介绍无机环境与异世界生物之间的关系。生物从无机环境中诞生,生物活动时刻改变着无机环境,而变化的无机环境又反作用于生物本身。如此循环往复,最终达到平衡。

本章涉及了一些生物学,尤其是生态学的内容,这可以帮助我们更完美地描绘异世界的“外貌”。

\section{生物的演化}
1859年达尔文的《物种起源》一书首次对生物演化机制做出了较为系统、完善的诠释,其要点可归纳为“变异和遗传、过度繁殖、生存斗争和适者生存”。达尔文认为过度繁殖使生存斗争加剧,自然选择使得能够适应环境的个体生存下来;个体可遗传的变异是生物演化的内在因素,而自然选择是生物演化的外在因素。环境和可遗传变异是选择与被选择的关系。

化石、胚胎学和比较解剖学以及细胞生物学和分子生物学分别为生物演化研究提供了直接、间接和微观证据,它们表明地球上的生物可能具有共同的祖先。

\section{生态系统}
\textbf{生态系统}(ecosystem)是指在一个特定环境内,由生物群落与非生物因素相互作用而形成的、能够自我维持的整体。

生态系统的范围没有固定的大小。小到一个池塘,大到沙漠、草原。行星上最大的生态系统叫\textbf{生物圈}(biosphere),包含了其上所有生态系统,是全部生物及其环境的总和。

\subsection{种群和群落}
在生态学上,\textbf{种群}(population)是指在一定空间范围内同时生活的同种生物的全部个体。在同一时间内,聚集在一定区域内各种生物种群的集合则称为\textbf{群落}(community)。

\subsection{食物链和食物网}
\textbf{食物链}(food chain)是在生态系统中不同营养级的生物通过捕食与被捕食的关系而建立起来的关系。在生态学中,食物链能代表生物种群间的物质循环和能量流动的情况。任何一种生物的增加或减少都会影响到其它生物的生存。不同的食物链彼此相互交错连结而形成的复杂网状结构称为\textbf{食物网}(food web)。

\section{演替}
\textbf{演替}(ecological succession)是一个群落的物种结构随时间变化的过程,是由低级到高级、由简单到复杂、一个阶段接着一个阶段、一个群落代替另一个群落的自然演变现象。演替可分为初生演替和次生演替两大类。

\subsection{初生演替}
如果一个地区从未被生物定居过(或者彻底清除了一切生物),生物从无开始的演替,称为\textbf{初生演替}(primary succession)。这是一个“巨岩变细石,石上长青苔”的过程,会经历极其漫长的岁月。

\begin{figure}[htbp]
    \centering
    \includegraphics[width=400pt]{images/primary-succession.png}
    \caption{初生演替的各个阶段}
    \label{fig:primary-succession}
\end{figure}

如图 \ref{fig:primary-succession} 展示了初生演替的7个阶段。

遥想亿万斯年以前,陆地上全是一望无际的、光秃秃的岩石。这一时期(阶段I)岩石上没有定居任何生命,称为裸岩阶段。

接着,地衣、苔藓、藻类和真菌等\textbf{先锋物种}(pioneer species)出现了。它们和风、水等非生物因素加速岩石的风化、崩解,形成土壤以及其它重要物质。这一阶段(阶段II)称为先锋物种阶段。

随着土壤有机物含量的提高、土层增厚,土壤的保水能力和养分进一步提高,草本植物的种子终于开始萌发。它们不仅排挤掉了先锋物种的优势地位,互相之间的竞争压力也越来越大。从最初以一年生草本植物为主,逐渐过度到多年生草本植物。它们还为各种节肢动物提供栖息地。其中阶段III是一年生草本植物阶段,阶段IV称为多年生草本植物和草阶段。

\subsection{次生演替}
与初生演替不同,\textbf{次生演替}(secondary succession)始于一场突发事件(如野火、飓风),致使原有生态系统中种群数量极度减少,但土壤并未受到十分严重的破坏的环境中。而初生演替的群落则缺乏土壤。

\begin{figure}[htbp]
    \centering
    \includegraphics[width=400pt]{images/secondary-succession.png}
    \caption{次生演替的各个阶段}
    \label{fig:secondary-succession}
\end{figure}

如图 \ref{fig:secondary-succession} 展示了次生演替的8个阶段。
