\section*{前言}
亲爱的读者,你好!十分感谢你抽空阅读这本《异世界地理基础》。

各位如果高中选学过地理,或许还记得那本选择性必修一的名字叫做“自然地理基础”,里面讲的都是自然地理的内容。同样,本书叫做“异世界地理基础”,毫无疑问,肯定会详细的介绍一下异世界地理。唯一不同的区别就是官方教科书讲的是地球,而本书讲的是异世界(指笔者用柏林噪声生成的或是随便画出来的)。

笔者想写这本书的目的仅仅是为了复习高中学过的所有地理知识,顺便从一个全新的视角来审视它们。在标题中使用“异世界”这个词只是觉得它作为定语很不错。

本书主要分成了如下三部分:
\begin{description}
    \item[天体] 简要介绍异世界的行星运动规律
    \item[自然] 介绍异世界的自然地理规律
    \item[人文] 介绍自然对人类活动产生的影响
\end{description}

本书以\LaTeX 编写,采用CC BY-SA 4.0协议进行授权。

\begin{flushright}
    2024年4月
\end{flushright}
