\section*{前言}
亲爱的读者,你好!十分感谢你抽空阅读这本《异世界地理基础》。

各位如果高中选学过地理,或许还记得那本选择性必修一的名字叫做“自然地理基础”,里面讲的都是自然地理的内容。同样,本书叫做“异世界地理基础”,毫无疑问,肯定会详细的介绍一下异世界地理。唯一不同的区别就是官方教科书讲的是地球,而本书主要讲的是异世界(指笔者用柏林噪声生成的或是随便画出来的)。

笔者想写这本书的目的仅仅是为了回顾高中学过的所有地理知识(以及部分物理、生物知识),顺便从一个全新的视角来审视它们。在标题中使用“异世界”这个词只是觉得使用它作为定语很不错。

本书主要分成了如下四章:
\begin{description}
    \item[行星] 简要介绍异世界的行星运动规律
    \item[自然] 介绍异世界的自然地理规律
    \item[生物] 简要介绍异世界的自然环境与生物的关系
    \item[人类] 简要介绍异世界中自然环境与人类活动的关系
\end{description}

本书以\LaTeX 编写,所有的文字及图像都在CC BY-SA 4.0协议之条款下提供。你可以在GitHub\footnote{\url{https://github.com/zhengxyz123/isekai-geo}}上获得本书的所有源代码。

\begin{flushright}
    2024年4月
\end{flushright}
